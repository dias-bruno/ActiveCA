\documentclass[Royal,times,sageh]{sagej}

\usepackage{moreverb,url,natbib, multirow, tabularx}
\usepackage[colorlinks,bookmarksopen,bookmarksnumbered,citecolor=red,urlcolor=red]{hyperref}



% tightlist command for lists without linebreak
\providecommand{\tightlist}{%
  \setlength{\itemsep}{0pt}\setlength{\parskip}{0pt}}



\usepackage{booktabs}
\usepackage{longtable}
\usepackage{array}
\usepackage{multirow}
\usepackage{wrapfig}
\usepackage{float}
\usepackage{colortbl}
\usepackage{pdflscape}
\usepackage{tabu}
\usepackage{threeparttable}
\usepackage{threeparttablex}
\usepackage[normalem]{ulem}
\usepackage{makecell}
\usepackage{xcolor}


\begin{document}


\setcitestyle{aysep={,}}

\title{ActiveCA: Time Use Data from the General Social Survey of Canada
to Study Active Travel}

\runninghead{}

\author{Anon1\affilnum{}, Anon2\affilnum{}, Anon3\affilnum{}}

\affiliation{\affilnum{}{}}



\begin{abstract}
This paper describes the open data product \texttt{ActiveCA}, with
Canadian time use data about active travel. \texttt{ActiveCA} is an
\texttt{R} data package that contains analysis-ready data from cycles
General Social Survey cicles regarding active travel in Canada. The
package provides data set detailing active trip origins, destinations,
and duration, covering a wide variety of locations, such as home, work
or school, libraries, museums, restaurants, bars, sports centers, health
clinics, place of worship, and others. The package also details the
respondent's region characteristics, specifying wheter they live in a
metropolitan area and their province of residency. \texttt{ActiveCA}
also provides pre-estimated impedance functions for active travel for
Canadian Metropolitan Area and Census Agglomerations, which can be used
in accessibility analysis to calculate the cost of travel between
different locations. The package will continue to expand with
contributions from the authors and the broader community through
requests in the future. \texttt{ActiveCA} is freely accessible for
exploration and download from the associated Github repository, where
the documentation and code involved in creating and manipulating data
are detailed.
\end{abstract}

\keywords{Active; mobility; walking; cycling; travel time; impedance;
transportation; R}

\maketitle

\hypertarget{introduction}{%
\section{Introduction}\label{introduction}}

This paper presents the open data product \texttt{ActiveCA}. Open data
products (ODPs) are the outcome of a transparent process that transforms
raw data (open or not) into analysis-ready data, in which all stages of
development follow open principles \citep{arribas-bel2021}. ODPs differ
from general open data due to their high utility, added value and open
availability.

\texttt{ActiveCA} is an \texttt{R} data package that provides
analysis-ready data from the General Social Survey of Canada Cycles 2
(1986), 7 (1992), 12 (1998), 19 (2005), 24 (2010), and 29 (2015)
\citep{statisticscanada2024} regarding active travel. These data can be
used to study walking and cycling in Canada for various purposes over
time, and to calculate impedance functions for accessibility analysis.
The package provides pre-estimated impedance functions for cycling and
walking trips for Census Agglomerations and Census Metropolitan Areas.

To create this \texttt{R} data package, we collected, cleaned and
processed the Time Use collections from the GSS surveys to make them
ready for analysis. An \texttt{R} data package contains code, data and
documentation in a standardized format that can be installed by
\texttt{R} users via a centralized software repository, such as CRAN
(Comprehensive R Archive Network) and GitHub. Although the GSS surveys
are publicly sourced managed by Statistic Canada
\citep{statisticscanada2024}, preparing them for analysis can be
time-consuming, tedious and perhaps not even possible for those who try,
due to a lack of documentation or prior knowledge.

The aim of this paper is to walk readers through the data sets and
invite others to experiment in its uses and applications.
\texttt{ActiveCA} is freely available on GitHub for all to install and
freely use in the spirit of open and reproducible research. In addition
of enabling to obtain impedance functions, \texttt{ActiveCA} can be
adopted in various applications that even go beyond the range of
possibilities we have imagined. Not only the data, but also all the code
documenting the processing methodology is available for consultation and
evaluation in its repository. This package contributes to reducing the
barrier to using the information contained in GSS surveys to provide
data-driven decisions in transportation analysis.

\hypertarget{general-social-survey-gss-collection}{%
\section{General Social Survey (GSS)
collection}\label{general-social-survey-gss-collection}}

Statistics Canada \citeyearpar{statisticscanada2024} conducts GSS
surveys to obtain data on social trends to track changes in Canadians'
living conditions and well-being over time. This survey is used to
understand how citizens spend and manage their time and what factors
contribute to their happiness and stress. Created in 1985, the survey is
part of a series of independent, annual, and cross-sectional surveys.

In addition to the main topic, each GSS cycle includes new content that
addresses emerging and policy-relevant issues. Every five to seven
years, the Time Use Surveys \citep{statisticscanada2022} collect data on
respondents' participation and time spent on a wide range of everyday
activities using a 24-hour retrospective diary, with information on the
location of these activities (e.g.~at home, at work, etc.) and, for
non-personal activities, the people who were present with the respondent
at the time of the activity. In addition, time-use surveys also cover
topics related to leisure time, work-life balance, health, commuting,
culture and sports, and many others.

The most recent time use survey was carried out in 2022
\citep{wray2024}. However, the 2022 dataset has not been fully published
and, because of this, our analysis focused on the surveys from 1986 to
2015 (1986, 1992, 1998, 2005, 2010 and 2015). Time Use surveys are
composed of two data sets, the main one and the episode file, explained
in the following subsections.

\hypertarget{the-main-file}{%
\subsection{The Main File}\label{the-main-file}}

The main file compiles an large array of aggregated data, summarizing
the answers to the questionnaire and derived variables that summarize
the respondents' time use across different activities, locations, and
social interactions. This file documents the time and duration that
respondents allocate to each activity and location. The Main File
provides a overview of daily routines and social dynamics, not focusing
on individual activity episodes. Additionally, this file categorizes
activities into bigger groups and subcategories, facilitating the data's
analytical utility with additional metrics such as total transit time,
duration spent with household members, and counts of activities and
episodes.

\hypertarget{the-episode-file}{%
\subsection{The Episode File}\label{the-episode-file}}

The episode file records detailed data for each activity episode
reported by respondents. The entry includes the start and end times,
duration, location, and accompanying social context, informing when and
where activities occurred and with whom. The file distinguishes itself
by focusing on individual episodes rather than respondents, with the
data structured around the numerous activity instances that compose a
day of the respondent. Although respondent-specific characteristics are
not included within the episode file, it is possible to link the main
file and the episode file by using an identifiable variable present in
both data sets.

\hypertarget{methodology}{%
\section{Methodology}\label{methodology}}

For each selected cycle of the GSS surveys, we reviewed the episode
files to identify cases with activities listed as walking or cycling,
selecting the activities immediately before and after the mobility
episode. After that, we labeled the code variables with their
appropriate descriptions, identifying each origin and destination,
transportation mode, as well the province and urban classification of
the respondent's residency.

Additionally, for Census Agglomerations and Census Metropolitan Areas we
pre-estimated impedance functions for cycling and walking trips.
Impedance functions reveals important information about the travel
behavior of the population, by describing the relationship between the
population at an origin and their likelihood or ability to travel to
specific destinations to access opportunities \citep{soukhov2024}.

Impedance functions are commonly used in accessibility analysis to
calculate the cost of travel between different locations
\citep{hansen1959, páez2012, palacios2022}. However, these functions
need to be calibrated in order to accurately represent travel behavior.
One effective way of calibrating an impedance function is by using the
trip length distribution (TLD) obtained from origin-destination data
\citep{soukhov2024}. The TLD represents the probability that a certain
proportion of trips will be made at a specific cost, such as travel
time. In this data set, low travel times are associated with a higher
proportion of trips, while high travel times are associated with a lower
proportion of trips.

For each combination of year, destination, and transportation mode
(walking or cycling), we fitted the most suitable impedance function
based on empirical data from the GSS surveys. We used the
\texttt{fitdistrplus} package \citep{mullerdutang2015} to estimate the
functions, selecting the distribution with the lowest Akaike information
criterion (AIC) among exponential, gamma, log-normal, normal, and
uniform types.

\hypertarget{results}{%
\section{Results}\label{results}}

\hypertarget{descriptive-statistics}{%
\subsection{Descriptive statistics}\label{descriptive-statistics}}

Considering GSS Cycles analyzed, we identified 21748 episodes that
recorded active travel episodes, with trip duration ranging from 0 to
900 minutes, to twelve different destinations. \texttt{ActiveCA}
includes all these episodes ready for analysis. Table \ref{tab:table-01}
presents descriptive statistics on walking and cycling trips between
1986 and 2015, including metrics such as the count of recorded trips
(count), and measures of trip duration in minutes: maximum (max), mean,
median, and minimum (min). The 1986 survey did not include bicycle
trips.

\begingroup\fontsize{10}{12}\selectfont

\begin{longtable}[t]{>{}llcccccc}
\caption{\label{tab:table-01}\label{tab:table-01}Descriptive statistics for episodes with active transport records}\\
\toprule
\multicolumn{2}{c}{ } & \multicolumn{6}{c}{Year} \\
\cmidrule(l{3pt}r{3pt}){3-8}
Mode & Statistic & 1986 & 1992 & 1998 & 2005 & 2010 & 2015\\
\midrule
 & Count & 4347 & 1500 & 1670 & 5533 & 4379 & 3251\\
\nopagebreak
 & Maximum & 660 & 300 & 255 & 515 & 480 & 900\\
\nopagebreak
 & Mean & 21 & 19 & 11 & 12 & 12 & 17\\
\nopagebreak
 & Median & 10 & 10 & 5 & 10 & 8 & 10\\
\nopagebreak
 & Minimum & 1 & 1 & 1 & 0 & 0 & 5\\
\nopagebreak
\multirow[t]{-6}{*}{\raggedright\arraybackslash \textbf{Walking}} & Standard deviation & 31 & 25 & 17 & 16 & 17 & 27\\
\cmidrule{1-8}\pagebreak[0]
 & Count &  & 135 & 119 & 333 & 236 & 245\\
\nopagebreak
 & Maximum &  & 240 & 90 & 180 & 153 & 120\\
\nopagebreak
 & Mean &  & 31 & 21 & 19 & 21 & 24\\
\nopagebreak
 & Median &  & 20 & 15 & 15 & 15 & 15\\
\nopagebreak
 & Minimum &  & 5 & 2 & 1 & 1 & 5\\
\nopagebreak
\multirow[t]{-6}{*}{\raggedright\arraybackslash \textbf{Cycling}} & Standard deviation &  & 36 & 18 & 18 & 23 & 20\\
\bottomrule
\end{longtable}
\endgroup{}

Table \ref{tab:table-01} shows that the median values for walking trips
range between 5 and 10 minutes, while cycling trips have a consistent
median of 15 minutes since 1998. The table also highlights very high
maximum values, particularly for walking trips, with recorded episodes
exceeding 4 hours in all cases.

Table \ref{tab:table-02} and \ref{tab:table-03} provide descriptive
statistics for the two modes of transportation, split by destination
categories, from 1986 to 1998 and from 2005 to 2015, respectively. In
Table \ref{tab:table-02}, one can observed that in 1986 and 1992,
walking trips destined for \texttt{home} had the highest medians.
However, by 1998, the highest medians shifted to trips to
\texttt{work\ or\ school}, a transition that also occurred for cycling
trips between 1992 and 1998. Table \ref{tab:table-03} indicates that the
median duration for trips to \texttt{home} and \texttt{work\ or\ school}
remained at 10 minutes.

\begingroup\fontsize{6}{8}\selectfont

\begin{ThreePartTable}
\begin{TableNotes}
\item \textit{Note: } 
\item * The symbols used in this table represent the following: 'Min' denotes the minimum time to reach the destination; 'Max' denotes the maximum time to reach the destination; '(\%)' indicates a percentage of the total time to reach the destination; 'Med' refers to the median time to reach the destination
\end{TableNotes}
\begin{longtable}[t]{ccccc>{}c|ccc>{}c|cccc}
\caption{\label{tab:table-02}\label{tab:table-02}Comparison of travel statistics by mode and destination: 1986, 1992, 1998}\\
\toprule
\multicolumn{2}{c}{ } & \multicolumn{4}{c}{1986} & \multicolumn{4}{c}{1992} & \multicolumn{4}{c}{1998} \\
\cmidrule(l{3pt}r{3pt}){3-6} \cmidrule(l{3pt}r{3pt}){7-10} \cmidrule(l{3pt}r{3pt}){11-14}
\multicolumn{1}{c}{\textbf{Destination}} & \multicolumn{1}{c}{\textbf{Mode*}} & \multicolumn{1}{c}{\textbf{Min*}} & \multicolumn{1}{c}{\textbf{Med*}} & \multicolumn{1}{c}{\textbf{Max*}} & \multicolumn{1}{c}{\textbf{(\%)*}} & \multicolumn{1}{c}{\textbf{Min}} & \multicolumn{1}{c}{\textbf{Med}} & \multicolumn{1}{c}{\textbf{Max}} & \multicolumn{1}{c}{\textbf{(\%)}} & \multicolumn{1}{c}{\textbf{Min}} & \multicolumn{1}{c}{\textbf{Med}} & \multicolumn{1}{c}{\textbf{Max}} & \multicolumn{1}{c}{\textbf{(\%)}}\\
\midrule
 & Home &  &  &  &  & 5 & 20 & 240 & 55.6 & 2 & 15.0 & 90 & 52.9\\
\nopagebreak
 & Other's home &  &  &  &  & 5 & 10 & 145 & 18.5 & 2 & 10.0 & 80 & 17.6\\
\nopagebreak
\multirow[t]{-3}{*}{\centering\arraybackslash Cycling} & Work or school &  &  &  &  & 5 & 15 & 45 & 25.9 & 5 & 20.0 & 75 & 29.4\\
\cmidrule{1-14}\pagebreak[0]
 & Home & 1 & 15 & 330 & 46.4 & 1 & 10 & 300 & 59.5 & 1 & 5.0 & 255 & 51.6\\
\nopagebreak
 & Other's home & 1 & 10 & 660 & 42.3 & 1 & 5 & 135 & 21.3 & 1 & 5.0 & 120 & 28.1\\
\nopagebreak
\multirow[t]{-3}{*}{\centering\arraybackslash Walking} & Work or school & 1 & 10 & 450 & 11.3 & 2 & 10 & 60 & 19.2 & 1 & 6.5 & 75 & 20.4\\
\bottomrule
\insertTableNotes
\end{longtable}
\end{ThreePartTable}
\endgroup{}

\begingroup\fontsize{6}{8}\selectfont

\begin{ThreePartTable}
\begin{TableNotes}
\item \textit{Note: } 
\item * The symbols used in this table represent the following: 'Min' denotes the minimum time to reach the destination; 'Max' denotes the maximum time to reach the destination; '(\%)' indicates a percentage of the total time to reach the destination; 'Med' refers to the median time to reach the destination
\end{TableNotes}
\begin{longtable}[t]{ccccc>{}c|ccc>{}c|cccc}
\caption{\label{tab:table-03}\label{tab:table-03}Comparison of travel statistics by mode and destination: 2005, 2010, 2015}\\
\toprule
\multicolumn{2}{c}{ } & \multicolumn{4}{c}{2005} & \multicolumn{4}{c}{2010} & \multicolumn{4}{c}{2015} \\
\cmidrule(l{3pt}r{3pt}){3-6} \cmidrule(l{3pt}r{3pt}){7-10} \cmidrule(l{3pt}r{3pt}){11-14}
\multicolumn{1}{c}{\textbf{Destination}} & \multicolumn{1}{c}{\textbf{Mode}} & \multicolumn{1}{c}{\textbf{Min*}} & \multicolumn{1}{c}{\textbf{Med*}} & \multicolumn{1}{c}{\textbf{Max*}} & \multicolumn{1}{c}{\textbf{(\%)*}} & \multicolumn{1}{c}{\textbf{Min}} & \multicolumn{1}{c}{\textbf{Med}} & \multicolumn{1}{c}{\textbf{Max}} & \multicolumn{1}{c}{\textbf{(\%)}} & \multicolumn{1}{c}{\textbf{Min}} & \multicolumn{1}{c}{\textbf{Med}} & \multicolumn{1}{c}{\textbf{Max}} & \multicolumn{1}{c}{\textbf{(\%)}}\\
\midrule
 & Cultural venues & 10 & 12.5 & 15 & 0.6 & 10 & 25 & 30 & 1.3 & 15 & 15.0 & 15 & 0.8\\
\nopagebreak
 & Grocery store & 2 & 10.0 & 30 & 10.2 & 5 & 10 & 75 & 8.9 & 5 & 15.0 & 80 & 6.5\\
\nopagebreak
 & Health clinic &  &  &  &  &  &  &  &  & 10 & 15.0 & 90 & 2.0\\
\nopagebreak
 & Home & 1 & 15.0 & 180 & 48.9 & 1 & 15 & 135 & 50.4 & 5 & 20.0 & 120 & 46.9\\
\nopagebreak
 & Neighbourhood &  &  &  &  &  &  &  &  & 10 & 30.0 & 45 & 1.2\\
\nopagebreak
 & Other's home & 1 & 15.0 & 35 & 9.0 & 5 & 10 & 45 & 9.3 & 5 & 15.0 & 40 & 5.3\\
\nopagebreak
 & Outdoors & 5 & 15.0 & 45 & 6.0 & 3 & 10 & 115 & 3.8 & 15 & 20.0 & 30 & 1.2\\
\nopagebreak
 & Place of worship & 20 & 20.0 & 20 & 0.3 &  &  &  &  & 15 & 15.0 & 15 & 0.4\\
\nopagebreak
 & Restaurant & 5 & 20.0 & 35 & 3.0 & 10 & 15 & 153 & 2.1 & 10 & 17.5 & 60 & 4.1\\
\nopagebreak
 & Sport area &  &  &  &  &  &  &  &  & 10 & 15.0 & 15 & 2.9\\
\nopagebreak
\multirow[t]{-11}{*}{\centering\arraybackslash Cycling} & Work or school & 1 & 15.0 & 90 & 21.9 & 1 & 15 & 100 & 24.2 & 5 & 15.0 & 120 & 28.6\\
\cmidrule{1-14}\pagebreak[0]
 & Business &  &  &  &  &  &  &  &  & 5 & 10.0 & 30 & 0.2\\
\nopagebreak
 & Cultural venues & 5 & 12.5 & 40 & 0.6 & 2 & 10 & 40 & 0.7 & 5 & 10.0 & 40 & 1.5\\
\nopagebreak
 & Grocery store & 1 & 10.0 & 90 & 12.5 & 1 & 8 & 105 & 13.2 & 5 & 10.0 & 130 & 11.8\\
\nopagebreak
 & Health clinic &  &  &  &  &  &  &  &  & 5 & 10.0 & 130 & 1.0\\
\nopagebreak
 & Home & 0 & 10.0 & 515 & 44.4 & 0 & 10 & 270 & 43.6 & 5 & 10.0 & 900 & 45.3\\
\nopagebreak
 & Neighbourhood &  &  &  &  &  &  &  &  & 5 & 10.0 & 60 & 2.1\\
\nopagebreak
 & Other's home & 1 & 5.0 & 300 & 11.7 & 0 & 5 & 140 & 11.3 & 5 & 10.0 & 120 & 7.3\\
\nopagebreak
 & Outdoors & 1 & 5.0 & 295 & 3.6 & 0 & 10 & 480 & 5.2 & 5 & 10.0 & 135 & 2.8\\
\nopagebreak
 & Place of worship & 1 & 10.0 & 30 & 0.8 & 1 & 8 & 60 & 0.9 & 5 & 15.0 & 45 & 1.1\\
\nopagebreak
 & Restaurant & 0 & 5.0 & 85 & 9.3 & 1 & 5 & 153 & 10.0 & 5 & 10.0 & 120 & 8.4\\
\nopagebreak
 & Sport area &  &  &  &  &  &  &  &  & 5 & 10.0 & 45 & 3.3\\
\nopagebreak
\multirow[t]{-12}{*}{\centering\arraybackslash Walking} & Work or school & 0 & 10.0 & 175 & 17.1 & 0 & 10 & 150 & 15.0 & 5 & 10.0 & 190 & 15.1\\
\bottomrule
\insertTableNotes
\end{longtable}
\end{ThreePartTable}
\endgroup{}

\texttt{ActiveCA} also enables visual analysis of active travel in
Canada using traditional exploratory data analysis techniques. Figures
\ref{fig:figure-01} and \ref{fig:figure-02} show walking and cycling
trips from 1992 and 2015 through heat maps. These maps use color
gradients to represent the percentage of trips between various origins
and destinations, with darker colors indicating higher percentages and
lighter colors representing less frequent routes. To avoid overwhelming
the reader, we omitted the heat maps for the other years analyzed.

In 1992, walking trips with \texttt{home} as both the origin and
destination made up the majority, accounting for about 30\% of all
walking trips. These trips often involved leisure activities, like short
walks or dog walking. Following this, trips from \texttt{home} to
\texttt{work\ or\ school} comprised 18\% of walking trips. Overall,
\texttt{home} emerged as a crucial hub, either as an origin or
destination, with only 5\% of trips not involving \texttt{home.} By
2015, \texttt{home} remained a significant node, but new locations
distributed the proportion of trips to areas not considered in 1992. In
2015, the highest proportion of trips were from \texttt{home} to
\texttt{work\ or\ school} (12\%) and vice versa (11\%). \texttt{home} to
\texttt{home} accounted for 8\% of trips, and \texttt{grocery\ stores}
became a notable destination for those leaving \texttt{home} (6\%),
surpassing trips to \texttt{other\textquotesingle{}s\ home} (4\%).

\begin{figure}

{\centering \includegraphics[width=1\linewidth]{Manuscript-figures/walking_hm_fig} 

}

\caption{Percentage of walking trips categorized by origin and destination}\label{fig:figure-01}
\end{figure}

For cycling trips, Figure \ref{fig:figure-02}, shows that in 1992, when
this mode of transportation was first included as an activity, the
majority of trips were from \texttt{home} to \texttt{work\ or\ school},
accounting for about 25\% of cases. This pattern remained in 2015, with
these trips representing 30\% of the cases. However, a notable change
occurred in \texttt{home} to \texttt{home} trips, which decreased
significantly from 19\% in 1992 to 5\% in 2015.

\begin{figure}

{\centering \includegraphics[width=1\linewidth]{Manuscript-figures/cycling_hm_fig} 

}

\caption{Percentage of cycling trips categorized by origin and destination}\label{fig:figure-02}
\end{figure}

\hypertarget{pre-estimated-impedance-functions-for-active-travel-in-canada}{%
\subsection{Pre-estimated impedance functions for active travel in
Canada}\label{pre-estimated-impedance-functions-for-active-travel-in-canada}}

\texttt{ActiveCA} includes a total of 64 pre-estimated impedance
functions for cycling and walking trips for Census Agglomerations and
Census Metropolitan Areas. Table \ref{tab:table-04} shows the best
impedance functions for walking trips to \texttt{Work\ or\ school}
across survey years, where all distributions, except for a gamma
function in 1998, are log-normal.

\begin{table}
\centering
\caption{\label{tab:table-04}\label{table-04}Impedance functions and AIC for `Walking` trips considering `Work or school` as destination.}
\caption{\label{tab:table-04}\label{table-04}Impedance functions and AIC for `Cycling` trips considering `Work or school` as destination.}
\centering
\resizebox{\ifdim\width>\linewidth\linewidth\else\width\fi}{!}{
\fontsize{10}{12}\selectfont
\begin{threeparttable}
\begin{tabular}[t]{lrlrrrr}
\toprule
\multicolumn{1}{c}{\textbf{Mode}} & \multicolumn{1}{c}{\textbf{Year}} & \multicolumn{1}{c}{\textbf{Impedance function}} & \multicolumn{1}{c}{\textbf{Parameter 1*}} & \multicolumn{1}{c}{\textbf{Parameter 2*}} & \multicolumn{1}{c}{\textbf{AIC}} & \multicolumn{1}{c}{\textbf{Count}}\\
\midrule
 & 1992 & Gamma & 3.00 & 0.17 & 433582 & 19\\

 & 1998 & Gamma & 3.37 & 0.10 & 481536 & 19\\

 & 2005 & Lognormal & 2.93 & 0.70 & 888655 & 64\\

 & 2010 & Lognormal & 2.65 & 0.77 & 1292760 & 53\\

\multirow[t]{-5}{*}{\raggedright\arraybackslash Cycling} & 2015 & Lognormal & 3.03 & 0.41 & 1162876 & 63\\
\cmidrule{1-7}
 & 1992 & Lognormal & 2.38 & 0.70 & 2319400 & 113\\

 & 1998 & Gamma & 1.23 & 0.09 & 2318752 & 109\\

 & 2005 & Lognormal & 2.13 & 0.79 & 8182691 & 724\\

 & 2010 & Lognormal & 2.21 & 0.78 & 7917431 & 494\\

\multirow[t]{-5}{*}{\raggedright\arraybackslash Walking} & 2015 & Lognormal & 2.55 & 0.64 & 6612061 & 407\\
\bottomrule
\end{tabular}
\begin{tablenotes}
\item \textit{Note: } 
\item *For lognormal distributions, 'Parameter 1' and 'Parameter 2' refer to the mean and standard deviation of the distribution on the logarithmic scaler, respectively. For gamma distribution, 'Parameter 1' and 'Parameter 2' refer to the rate and shape of the distribution, respectively. `AIC` means Akaike information criterion.
\end{tablenotes}
\end{threeparttable}}
\end{table}

Figure \ref{fig:figure-03} illustrates these fitted functions (red line)
alongside histograms of empirical travel times (grey bars). As for the
others pre-estimated functions, these examples allow to calculate
gravity-based accessibility measures for active travel across various
destinations and temporal scales in Canadian urban areas.

\begin{figure}

{\centering \includegraphics[width=1\linewidth]{Manuscript-figures/impf_Work or school} 

}

\caption{Empirical data and impedance functions fitted for `work or school` as destination.}\label{fig:figure-03}
\end{figure}

\hypertarget{concluding-remarks}{%
\section{Concluding remarks}\label{concluding-remarks}}

This article presents \texttt{ActiveCA}, an open data product that
provides analysis-ready data from Cycles 2 (1986), 7 (1992), 12 (1998),
19 (2005), 24 (2010), and 29 (2015) of the GSS surveys on active travel
in Canada. In the form of an \texttt{R} data package, \texttt{ActiveCA}
was developed after collecting, cleaning, and processing the survey
data, providing information on origins, destinations, and duration of
active travel, as well other information. Additionally, the package
includes a series of pre-estimated impedance functions for walking and
cycling trips, considering various destinations and time periods, for
Canadian Metropolitan and Census Agglomeration areas.

The value of \texttt{ActiveCA} lies in its transparency, accessibility,
and open infrastructure, which facilitates the addition of complementary
data sets in the future. R users can seamlessly explore GSS walking and
cycling episodes along with calibrated impedance functions, with the
option to suggest enhancements to the package as needed. This article
adopts the structure proposed by Anastasia and Páez
\citeyearpar{soukhov2023}, whose work provided essential guidance for
the creation of this package. Similarly, we aim to contribute to the
academic community by promoting transparent research practices that
encourage replication and innovation in related fields. We believe that
\texttt{ActiveCA} will serve as a basis for further research on GSS and
for the integration of additional data by the authors or the wider open
source community.

\hypertarget{declaration-of-conflicting-interests}{%
\section{Declaration of Conflicting
Interests}\label{declaration-of-conflicting-interests}}

The author(s) declared no potential conflicts of interest with respect
to the research, authorship, and/or publication of this article.

\hypertarget{funding}{%
\section{Funding}\label{funding}}

The author(s) disclosed receipt of the following financial support for
the research, authorship, and/or publication of this article: This work
was supported by the Social Sciences and Humanities Research Council of
Canada (\emph{More description about the funding source after the review
process}).

\hypertarget{orcid}{%
\section{ORCID}\label{orcid}}

Author 1

Author 2

Author 3

\hypertarget{data-availability-statement}{%
\section{Data availability
statement}\label{data-availability-statement}}

The \{ActiceCA\} R data package can be found and installed on Github
(\emph{link}).

\bibliographystyle{sageh}
\bibliography{bibfile.bib}


\end{document}
